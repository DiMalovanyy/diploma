\newpage
\chapter*{\textsc{висновки}}
\addcontentsline{toc}{chapter}{\textsc{висновки}}
В роботі "Гомоморфне шифрування для захисту даних в хмарних та туманних технологiях" досліджується гомоморфне шифрування, зокрема BGV (Brakerski-Gentry-Vaikuntanathan) схема. Гомоморфне шифрування є принципово новим підходом до захисту конфіденційних даних, який дозволяє виконувати обчислення над зашифрованими даними, зберігаючи їх у зашифрованому вигляді. Це забезпечує високий рівень конфіденційності та захисту інформації, що є особливо важливим у сферах, де зберігаються чутливі дані, наприклад, в банківській сфері.

У рамках дослідження проведена теоретична експертиза гомоморфного шифрування. Були вивчені математичні принципи, на яких ґрунтується гомоморфне шифрування, включаючи алгебраїчні структури та протоколи шифрування.

Для зрозуміння основних концепцій та технічних деталей гомоморфного шифрування було проаналізовано різні підходи, методи та алгоритми, які лежать в основі BGV схеми. Вивчення властивостей гомоморфного шифрування, таких як гомоморфність додавання та множення, а також операцій перетину та об'єднання, було проведено для оцінки його потенціалу в застосуванні до захисту даних.

Додатково, були досліджені сучасні протоколи та алгоритми, які дозволяють оптимізувати та поліпшити ефективність гомоморфного шифрування, зокрема у контексті обробки великих обсягів даних. Це включало аналіз методів оптимізації, таких як гомоморфна оцінка, техніки упаковки та інші методи зменшення обчислювальної складності.

В результаті теоретичної експертизи було отримано глибоке розуміння принципів гомоморфного шифрування, його потенціалу та обмежень. Це дозволило розробити клієнт-серверний застосунок банківської системи з використанням гомоморфного шифрування та бібліотеки HeLib. Теоретична експертиза була важливим етапом для успішної реалізації системи та її використання в практичних сценаріях.


У роботі було реалізовано клієнт-серверний застосунок банківської системи з використанням бібліотеки HeLib. У цій системі сервер зберігає інформацію про рахунки користувачів у зашифрованому форматі в базі даних. Клієнт може виконувати такі операції, як створення рахунків, додавання балансу, зняття балансу та отримання інформації про рахунок. Всі ці операції відбуваються над зашифрованими даними без необхідності розшифрування їх на сервері, що забезпечує високий рівень безпеки.

Використання гомоморфного шифрування має свої обмеження та недоліки. Основним обмеженням є обчислювальна складність таких систем. Гомоморфне шифрування вимагає значних обчислювальних ресурсів, що може призвести до затримок у виконанні операцій та збільшення обсягу обробки даних. Крім того, розмір зашифрованих даних може бути більшим, ніж у випадку звичайного шифрування, що може вплинути на продуктивність системи.

Недоліком гомоморфного шифрування є також вразливість до атак, зокрема до криптоаналітичних методів, які можуть використовувати математичні властивості схеми для отримання доступу до зашифрованих даних. Пошук ефективних захистів та протоколів залишається активною областю дослідження.

Крім обмежень і недоліків, варто відзначити, що гомоморфне шифрування також вимагає спеціального розуміння та експертизи для його впровадження та використання. Розробка та підтримка систем, які використовують гомоморфне шифрування, можуть вимагати високо кваліфікованого персоналу, який розуміє принципи шифрування та математичні основи, на яких воно ґрунтується.

Усупереч обмеженням та недолікам, гомоморфне шифрування має значний потенціал для захисту даних у хмарних та туманних технологіях, де конфіденційні дані можуть бути оброблені без необхідності розкриття їх змісту. Подальше дослідження та розробка ефективних алгоритмів гомоморфного шифрування можуть сприяти розширенню його застосування та підвищенню безпеки обробки конфіденційної інформації.
