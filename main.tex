\documentclass[a4paper,14pt]{extreport} % Розмір паперу А4, шрифт 14 пунктів
\usepackage[T2A]{fontenc}
\usepackage[english,ukrainian]{babel}
\usepackage{ucs}
\usepackage{url}
\usepackage[utf8]{inputenc} % включємо кодування utf8 в *NIX (cp1251 в Windows)
\usepackage{amssymb,amsfonts,amsmath,mathtext,enumerate,float} % підключаємо пакети розширень
\usepackage{listings} % для вихідних кодів
\usepackage{alltt}  
\usepackage{algorithmic}
\usepackage{indentfirst} % для абзаців
\usepackage[pdftex]{graphicx}
\usepackage{titlesec}
\usepackage{amsthm}
\usepackage{etoolbox}
\usepackage{spverbatim}



\renewcommand{\rmdefault}{cmr}
\renewcommand{\sfdefault}{ftx}
\renewcommand{\ttdefault}{cmtt}

\makeatletter
\nocite{*}
\bibliographystyle{plain} % встановлюємо тип бібліографії

\renewcommand{\@biblabel}[1]{#1.} % змінюємо формат нумерації бібліографії на "цифра."
\makeatother

\usepackage{geometry} % Змінюємо поля сторінки
\geometry{left=3cm}
\geometry{right=1cm}
\geometry{top=2cm}
\geometry{bottom=2cm}

\usepackage{setspace} % інтерліньяж
\onehalfspacing

\addto\captionsukrainian{
  \renewcommand{\contentsname}{зміст}
  \renewcommand{\bibname}{перелік джерел}
}

\titleformat{\chapter}[display]
  {\normalfont\centering\Large\bfseries}
  {\chaptertitlename\ \thechapter}{0pt}{\Large\textsc}

% Змінюємо скрізь перелічення на наступні "цифра.цифра.":
\renewcommand{\theenumi}{\arabic{enumi}.} 	
\renewcommand{\labelenumi}{\arabic{enumi}.} 
\renewcommand{\theenumii}{.\arabic{enumii}.} 
\renewcommand{\labelenumii}{\arabic{enumi}.\arabic{enumii}.} 
\renewcommand{\theenumiii}{.\arabic{enumiii}.} 
\renewcommand{\labelenumiii}{\arabic{enumi}.\arabic{enumii}.\arabic{enumiii}.}

\righthyphenmin=2 % мінімальна к-ть символів для переносу
\sloppy

\theoremstyle{definition}
\newtheorem{definition}{Визначення}[section]
\apptocmd{\thebibliography}{\raggedright}{}{}
\newtheorem{theorem}{Теорема}[section]

\begin{document}

	\newcommand{\signature}{
{\footnotesize
\begin{tabular}{@{}p{1in}@{}}
    \hrulefill \\
    \vspace{-\baselineskip}
    \centering{(підпис)}
\end{tabular}}
}

\newpage
\begin{titlepage}
\begin{center}
\textbf{КИЇВСЬКИЙ НАЦІОНАЛЬНИЙ УНІВЕРСИТЕТ ІМЕНІ ТАРАСА ШЕВЧЕНКА}\\
Факультет комп’ютерних наук та кібернетики\\
Кафедра теоретичної кібернетики
\end{center}
\vspace{1em}
\begin{center}
\Large{\textbf{Кваліфікаційна робота}}\\
\normalsize{\textbf{На здобуття ступеня бакалавра}}\\
за спеціальністю 122 Комп’ютерні науки

\vspace{0.5em}
на тему:

\large{\textsc{\textbf{Гомоморфне шифрування для захисту даних в хмарних та туманних технологіях}}}
\end{center}
\vspace{1em}
\begin{flushleft}
Виконав студент 4-го курсу\\
Мальований Дмитро Борисович
\hspace{\fill}\signature\\
\vspace{0.5em}
Науковий керівник:\\
професор, доктор фіз-мат. наук\\
Пашко Анатолій Олексійович
\hspace{\fill}\signature\\
\end{flushleft}

\begin{flushleft}
    \hspace{18em}Засвідчую, що в цій роботі немає\\
    \hspace{15em}запозичень праць інших авторів без\\
    \hspace{15em}відповідних посилань.\\
    \vspace{0.3em}
    \hspace{15em}Студент\hspace{\fill}\signature\\
    \vspace{1em}
    \hspace{15em}Роботу розглянуто й допущено до захисту\\
    \hspace{15em}на засіданні кафедри\\
    \hspace{15em}теоретичної кібернетики\\
    \hspace{15em}"\vtop{\hsize=2em \hrulefill}"\vtop{\hsize=5em \hrulefill} 2023p\\
    \hspace{15em}протокол \No \vtop{\hsize=2em \hrulefill}\\
    \hspace{15em}Завідувач кафедри\\
    \hspace{15em}Крак Юрій Васильович\hspace{\fill}\signature\\
\end{flushleft}

\vspace{\fill}

\begin{center}
Київ - 2023
\end{center}				
\end{titlepage}
 		% Титулка
 	\newpage
\chapter*{реферат}
Обсяг роботи 53 сторінки, 8 зображень, 5 лістингів, 33 джерел посилань.
ШИФРУВАННЯ, ГОМОМОРФНЕ ШИФРУВАННЯ, ХМАРНІ ТА ТУМАННІ КОМУНІКАЦІЇ, БЕЗПЕКА ПЕРЕДАЧІ ДАНИХ,
ЗАХИСТ ДАНИХ, БЕЗПЕКА, БЕЗПЕКА ДАНИХ В БАНКІВСЬКІЙ СИСТЕМІ.

Об'єктом роботи є дослідження можливостей використання гомоморфного шифрування в хмарних
та туманних обчисленнях. Предметом роботи, є реалізація спрощеної банківської системи, для
демонстрації можливостей гомоморфного шифрування.

Метою роботи є дослідження технології повного гомоморфного шифрування в хмарних та 
туманних технологій.

Методи розроблення: дослідження гомоморфних схем, аналітичне дослідження алгоритмів над
схемою. Інструменти розроблення: Мова С++, Бібліотека HeLib з вільно поширюваною ліцензією
Apache 2.0, додаткові бібліотеки для зручної роботи з Json, обробки та ініціації TCP 
з'єднань, та інш. 

Результати роботи: Описана логіка криптографічних схем гомоморфного шифрування, проведений
аналітичний огляд існуючих реалізацій схем, наведені переваги та недоліки використання 
технології гомоморфного шифрування, реалізований та продемонстрований в роботі застосунок
спрощеної банківської системи з використанням FHE.

Технологія гомоморфного шифрування, може використовуватись в будь-якій сфері де потрібна
конфіденційність даних, зокрема вона дозволяє тримати їх приватними для сторони яка їх
обробляє, що забезпечує ще вищий рівень безпеки.
		% Реферат
 	\input{contents_page} 	% Зміст
    \newpage
\chapter*{скорочення та умовні позначення}
\addcontentsline{toc}{chapter}{\textsc{скорочення та умовні позначення}}

\textbf{\textsc{FHE}} -- Fully homomorphic encryption (Повне гомоморфне шифрування).

\textbf{\textsc{SHE}} -- Somewhat homomorphic encryption scheme

\textbf{\textsc{RSA}} -- Криптографічний алгоритм з відкритим ключем, який базується,
розрахунковій складності великих полупростих чисел.

\textbf{\textsc{PKI}} -- Public key infrastructure - набір інструментів які
використовують пару (приватний, публічний) ключ, та в якій між користувачами
передається тільки публічні ключі, залишаючи приватні анонімними.

\textbf{\textsc{Boolean Circuit}} -- Булева схема - це математична модель, що
використовується для представлення та обробки булевих функцій. Вона складається з логічних
елементів, які з'єднані між собою для виконання логічних операцій над двійковими входами
\(\{0,1\}\) та формування двійкових виходів. Схема складається з взаємопов'язаних логічних
елементів, таких як (AND, NOT, OR). 

\textbf{\textsc{FLT}} -- Мала теорема ферма \cite{Fermat}.

\textbf{\textsc{NAND gate}} -- NOT, AND операції за допомогою який може бути представлена
булева схема.
          % Скорочення та умовні позначення
 	
 	\newpage
\addcontentsline{toc}{chapter}{\textsc{вступ}}
\chapter*{\textsc{вступ}}


FHE або повне гомоморфне шифрування, це тип шифрування яке дозволяє виконувати розрахунки
на зашифрованих даних, не вимагаючи, щоб вони були розшифровані для цього. Результатом
розрахунків або ж гомоморфної операції над даними є зашифровані дані, які можуть бути
розшифровані ключем з тої ж самої пари, з якої вони були зашифровані.

Завдяки особливості виконувати операції над зашифрованими даними без попереднього
дешифрування, FHE стає гарним рішенням в задачах передачі даних в незахищених
середовищах, в не авторизованих середовищах, або при передачі над чутливих даних, які
не повинні бути видимі для сторони яка займається їх обробкою.

\subsection*{Оцінка сучасного стану об’єкта дослідження або розробки}
Вперше технологія FHE була запропонована в 1978 в році, але майже 30 років не було
авторитетних досліджень на цю тему і на той момент вже існувала система RSA, яка була
краща за багатьма параметрам. Починаючи з 2009 року дослідження та розробки на
тему гомоморфного шифрування дуже актуальні й розвиток цієї технології відбувається
надзвичайно швидко, покращуючи швидкість виконання операцій, швидкість дешифрування, та
розширюючи область застосування шляхом додавання більш комплексних гомоморфних операцій.
На цей час існує багато рішень на типові проблеми з використовуванням FHE, які
конкурують між собою в різних аспектах, та постійно розвиваються.

\subsection*{Актуальність роботи та підстави для її виконання}
Безпечність передачі даних в незахищених середовищах та авторизація отримувача були 
завжди дуже важливими, рідко хто нехтує цим, оскільки не хоче, щоб їх данні були
скомпрометовані або перехоплені. Окрім цього все частіше, за потребою складних
обчислень, користувачі звертаються до віддалених машин, також відомі як хмари. Звісно
кожен користувач хоче, щоб їх данні були захищенні під час передачі, та хоче бути
впевнений, що він передає данні саме туди, куди планував.

Для забезпечення вище описаних вимог, користувач використовує чинні технології, такі
як PKI. Єдина не вирішена проблема PKI або інших технологій, це вимога повного
дешифрування даних, це означає що при отриманні зловмисником доступу до хмари або
віддаленого сервера, у нього буде доступ до не зашифрованих даних. Хоча сучасні хмари
дуже добре захищенні, розраховувати на те що зловмисник не зможе отримати до них
доступ - не варто. 

Для розв'язання проблеми, яку не вирішує PKI, чудово підходить FHE, оскільки сервер,
зберігає і виконує операції над даними в зашифрованому вигляді, тому навіть якщо
зловмисник отримає доступ до сервера або хмари, отримати дані в нього не вдасться.

Звісно є і деякі обмеження у використанні FHE: по-перше, операції над даними
обов'язково повинні бути гомоморфні, по-друге, алгоритм застосування операції над
зашифрованими даними дуже повільний. Якщо задача вимагає обробку великої кількості
даних, або операція повинна бути не гомоморфна, то можливо краще подумати в сторону
застосування інших криптосистем.

\subsection*{Мета й завдання роботи}

Мета роботи дослідити інсуючі гомоморфні схеми, визначити їх криптографічну схему, обмеження та
можливі області застосування. Також, необхідно показати імплементації існуючих повних та
частково гомоморфних схем, описати їх вразливості, аналітично порівняти
описані схеми.

Також метою роботи є засвідчення того що гомоморфне шифрування застосоване до задач
безпечної передачі даних у хмарних та туманних технологіях. Завдання полягає в тому,
щоб показати теоретично та практично, що дані користувача можуть бути безпечно передані
та оброблені хмарою, без розкриття цих даних для хмари.

Також необхідно перевірити результати практичного використання FHE на коректність, та порівняти
накладні витрати, по часу та пам'яті, виконання операції над зашифрованими даними, та над не
зашифрованими.

\subsection*{Об’єкт і методи дослідження}
Для дослідження коректності та застосованості FHE і практичної реалізації системи з
використанням технології FHE, було вибрано клієнт-серверний застосунок, де сервер буде
виконувати роль хмари, та з'єднання клієнта з сервером відбувається в незахищеному середовищі.

Областю реалізації буде спрощена банківська система, де хмара буде виконувати роль банку,
який дозволяє користувачу додавати, знімати, та передивлятись свій баланс віртуальних грошей.
При цьому серверний застосунок повинен бути реалізований таким чином, що він не буде знати
нічого, ні про користувача, а ні про то скільки умовного баланса у певного клієнта. Для цього
він буде зберігати данні зашифровані FHE у внутрішній базі даних, та публічний ключ клієнта
для виконання гомоморфних операцій над даними.

Ця система повинна чудово показати всю силу гомоморфного шифрування: тільки клієнт, який
створив баланс за допомогою свого приватного ключа, буде мати можливість мати доступ до свого
балансу, як переглядати його, так і виконувати над ним певні операції. Всі інші учасники та 
користувачі системи не матимуть доступу до даних, що забезпечує їх повну безпеку.

Більш детально про об'єкти та методи дослідження буде описано в другому розділі роботи, 
фрагменти реалізації будуть наведені в додатках до роботи.

\subsection*{Можливі сфери застосування}
Гомоморфне шифрування може бути застосоване в будь-якій сфері де потрібна обробка
даних, та для виконання цієї задачі використовується віддалений сервер, або хмара.
Використання FHE, гарантує безпечну передачу та обробку без попереднього дешифрування
даних, але при цьому накладає обмеження на операцію обробки, яка повинна бути
гомоморфна, та значно знижує час обробки.

Більш детально ця тема буде розкрита в відповідному розділі, де будуть описані як
повноцінні області застосування FHE, так і використання FHE як інструмент для створення
більш комплексних криптосистем, та інструментів.


 			% Вступ
 
 	\newpage

\chapter{\textsc{тестовий розділ}}

\section{Походження}
контент1
\subsection{Тестова сабсекція}
контент сабсекції
\section{Поширення}
контент2
\section{Класифікація}
контент3
  		% Розділ 1
 	\newpage

\chapter{\textsc{Використання FHE в хмарних технологіях}}

\section{Бібліотека HeLib}
Для реалізації поставленої задачі буде використовуватись бібліотека гомоморфного шифрування
HeLib. HeLib була написана на С++ та реалізовує функціонал  Brakerski-Gentry-Vaikuntanathan 
(BGV), та Cheon-Kim-Kim-Song (CKKS) схем.

\subsection{Алгоритми над схемою}
 		% Розділ 2
 	\newpage
\chapter*{\textsc{висновки}}
\addcontentsline{toc}{chapter}{\textsc{висновки}}
В роботі "Гомоморфне шифрування для захисту даних в хмарних та туманних технологiях" досліджується гомоморфне шифрування, зокрема BGV (Brakerski-Gentry-Vaikuntanathan) схема. Гомоморфне шифрування є принципово новим підходом до захисту конфіденційних даних, який дозволяє виконувати обчислення над зашифрованими даними, зберігаючи їх у зашифрованому вигляді. Це забезпечує високий рівень конфіденційності та захисту інформації, що є особливо важливим у сферах, де зберігаються чутливі дані, наприклад, в банківській сфері.

У рамках дослідження проведена теоретична експертиза гомоморфного шифрування. Були вивчені математичні принципи, на яких ґрунтується гомоморфне шифрування, включаючи алгебраїчні структури та протоколи шифрування.

Для зрозуміння основних концепцій та технічних деталей гомоморфного шифрування було проаналізовано різні підходи, методи та алгоритми, які лежать в основі BGV схеми. Вивчення властивостей гомоморфного шифрування, таких як гомоморфність додавання та множення, а також операцій перетину та об'єднання, було проведено для оцінки його потенціалу в застосуванні до захисту даних.

Додатково, були досліджені сучасні протоколи та алгоритми, які дозволяють оптимізувати та поліпшити ефективність гомоморфного шифрування, зокрема у контексті обробки великих обсягів даних. Це включало аналіз методів оптимізації, таких як гомоморфна оцінка, техніки упаковки та інші методи зменшення обчислювальної складності.

В результаті теоретичної експертизи було отримано глибоке розуміння принципів гомоморфного шифрування, його потенціалу та обмежень. Це дозволило розробити клієнт-серверний застосунок банківської системи з використанням гомоморфного шифрування та бібліотеки HeLib. Теоретична експертиза була важливим етапом для успішної реалізації системи та її використання в практичних сценаріях.


У роботі було реалізовано клієнт-серверний застосунок банківської системи з використанням бібліотеки HeLib. У цій системі сервер зберігає інформацію про рахунки користувачів у зашифрованому форматі в базі даних. Клієнт може виконувати такі операції, як створення рахунків, додавання балансу, зняття балансу та отримання інформації про рахунок. Всі ці операції відбуваються над зашифрованими даними без необхідності розшифрування їх на сервері, що забезпечує високий рівень безпеки.

Використання гомоморфного шифрування має свої обмеження та недоліки. Основним обмеженням є обчислювальна складність таких систем. Гомоморфне шифрування вимагає значних обчислювальних ресурсів, що може призвести до затримок у виконанні операцій та збільшення обсягу обробки даних. Крім того, розмір зашифрованих даних може бути більшим, ніж у випадку звичайного шифрування, що може вплинути на продуктивність системи.

Недоліком гомоморфного шифрування є також вразливість до атак, зокрема до криптоаналітичних методів, які можуть використовувати математичні властивості схеми для отримання доступу до зашифрованих даних. Пошук ефективних захистів та протоколів залишається активною областю дослідження.

Крім обмежень і недоліків, варто відзначити, що гомоморфне шифрування також вимагає спеціального розуміння та експертизи для його впровадження та використання. Розробка та підтримка систем, які використовують гомоморфне шифрування, можуть вимагати високо кваліфікованого персоналу, який розуміє принципи шифрування та математичні основи, на яких воно ґрунтується.

Усупереч обмеженням та недолікам, гомоморфне шифрування має значний потенціал для захисту даних у хмарних та туманних технологіях, де конфіденційні дані можуть бути оброблені без необхідності розкриття їх змісту. Подальше дослідження та розробка ефективних алгоритмів гомоморфного шифрування можуть сприяти розширенню його застосування та підвищенню безпеки обробки конфіденційної інформації.
 		% Висновки
 	\newpage
\addcontentsline{toc}{chapter}{\textsc{перелік джерел}}
\bibliography{resources}
 		% Перелік джерел
 	\newpage
\chapter*{\textsc{додатки}}
\addcontentsline{toc}{chapter}{\textsc{додатки}}
контент
 		% Додатки
	
\end{document}
