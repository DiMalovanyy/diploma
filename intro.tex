\newpage
\addcontentsline{toc}{chapter}{\textsc{вступ}}
\chapter*{\textsc{вступ}}


FHE або повне гомоморфне шифрування, це тип шифрування яке дозволяє виконувати розрахунки
на зашифрованих даних, не вимагаючи, щоб вони були розшифровані для цього. Результатом
розрахунків або ж гомоморфної операції над даними є зашифровані дані, які можуть бути
розшифровані ключем з тої ж самої пари, з якої вони були зашифровані.

Завдяки особливості виконувати операції над зашифрованими даними без попереднього
дешифрування, FHE стає гарним рішенням в задачах передачі даних в незахищених
середовищах, в не авторизованих середовищах, або при передачі над чутливих даних, які
не повинні бути видимі для сторони яка займається їх обробкою.

\subsection*{Оцінка сучасного стану об’єкта дослідження або розробки}
Вперше технологія FHE була запропонована в 1978 в році, але майже 30 років не було
авторитетних досліджень на цю тему і на той момент вже існувала система RSA, яка була
краща за багатьма параметрам. Починаючи з 2009 року дослідження та розробки на
тему гомоморфного шифрування дуже актуальні й розвиток цієї технології відбувається
надзвичайно швидко, покращуючи швидкість виконання операцій, швидкість дешифрування, та
розширюючи область застосування шляхом додавання більш комплексних гомоморфних операцій.
На цей час існує багато рішень на типові проблеми з використовуванням FHE, які
конкурують між собою в різних аспектах, та постійно розвиваються.

\subsection*{Актуальність роботи та підстави для її виконання}
Безпечність передачі даних в незахищених середовищах та авторизація отримувача були 
завжди дуже важливими, рідко хто нехтує цим, оскільки не хоче, щоб їх данні були
скомпрометовані або перехоплені. Окрім цього все частіше, за потребою складних
обчислень, користувачі звертаються до віддалених машин, також відомі як хмари. Звісно
кожен користувач хоче, щоб їх данні були захищенні під час передачі, та хоче бути
впевнений, що він передає данні саме туди, куди планував.

Для забезпечення вище описаних вимог, користувач використовує чинні технології, такі
як PKI. Єдина не вирішена проблема PKI або інших технологій, це вимога повного
дешифрування даних, це означає що при отриманні зловмисником доступу до хмари або
віддаленого сервера, у нього буде доступ до не зашифрованих даних. Хоча сучасні хмари
дуже добре захищенні, розраховувати на те що зловмисник не зможе отримати до них
доступ - не варто. 

Для розв'язання проблеми, яку не вирішує PKI, чудово підходить FHE, оскільки сервер,
зберігає і виконує операції над даними в зашифрованому вигляді, тому навіть якщо
зловмисник отримає доступ до сервера або хмари, отримати дані в нього не вдасться.

Звісно є і деякі обмеження у використанні FHE: по-перше, операції над даними
обов'язково повинні бути гомоморфні, по-друге, алгоритм застосування операції над
зашифрованими даними дуже повільний. Якщо задача вимагає обробку великої кількості
даних, або операція повинна бути не гомоморфна, то можливо краще подумати в сторону
застосування інших криптосистем.

\subsection*{Мета й завдання роботи}

Мета роботи дослідити інсуючі гомоморфні схеми, визначити їх криптографічну схему, обмеження та
можливі області застосування. Також, необхідно показати імплементації існуючих повних та
частково гомоморфних схем, описати їх вразливості, аналітично порівняти
описані схеми.

Також метою роботи є засвідчення того що гомоморфне шифрування застосоване до задач
безпечної передачі даних у хмарних та туманних технологіях. Завдання полягає в тому,
щоб показати теоретично та практично, що дані користувача можуть бути безпечно передані
та оброблені хмарою, без розкриття цих даних для хмари.

Також необхідно перевірити результати практичного використання FHE на коректність, та порівняти
накладні витрати, по часу та пам'яті, виконання операції над зашифрованими даними, та над не
зашифрованими.

\subsection*{Об’єкт і методи дослідження}
Для дослідження коректності та застосованості FHE і практичної реалізації системи з
використанням технології FHE, було вибрано клієнт-серверний застосунок, де сервер буде
виконувати роль хмари, та з'єднання клієнта з сервером відбувається в незахищеному середовищі.

Областю реалізації буде спрощена банківська система, де хмара буде виконувати роль банку,
який дозволяє користувачу додавати, знімати, та передивлятись свій баланс віртуальних грошей.
При цьому серверний застосунок повинен бути реалізований таким чином, що він не буде знати
нічого, ні про користувача, а ні про то скільки умовного баланса у певного клієнта. Для цього
він буде зберігати данні зашифровані FHE у внутрішній базі даних, та публічний ключ клієнта
для виконання гомоморфних операцій над даними.

Ця система повинна чудово показати всю силу гомоморфного шифрування: тільки клієнт, який
створив баланс за допомогою свого приватного ключа, буде мати можливість мати доступ до свого
балансу, як переглядати його, так і виконувати над ним певні операції. Всі інші учасники та 
користувачі системи не матимуть доступу до даних, що забезпечує їх повну безпеку.

Більш детально про об'єкти та методи дослідження буде описано в другому розділі роботи, 
фрагменти реалізації будуть наведені в додатках до роботи.

\subsection*{Можливі сфери застосування}
Гомоморфне шифрування може бути застосоване в будь-якій сфері де потрібна обробка
даних, та для виконання цієї задачі використовується віддалений сервер, або хмара.
Використання FHE, гарантує безпечну передачу та обробку без попереднього дешифрування
даних, але при цьому накладає обмеження на операцію обробки, яка повинна бути
гомоморфна, та значно знижує час обробки.

Більш детально ця тема буде розкрита в відповідному розділі, де будуть описані як
повноцінні області застосування FHE, так і використання FHE як інструмент для створення
більш комплексних криптосистем, та інструментів.


