\newpage
\addcontentsline{toc}{chapter}{\textsc{вступ}}
\chapter*{\textsc{вступ}}


FHE або повне гомоморфне шифрування, це тип шифрування яке дозволяє виконувати розрахунки
на зашифрованих даних, не вимагаючи, щоб вони були розшифровані для цього. Результатом
розрахунків або ж гомоморфної операції над даними є зашифровані дані, які можуть бути
розшифровані ключем з тої ж самої пари, з якої вони були зашифровані.

Завдяки особливості виконувати операції над зашифрованими даними без попереднього
дешифрування, FHE стає гарним рішенням в задачах передачі даних в незахищених
середовищах, в не авторизованих середовищах, або при передачі над чутливих даних, які
не повинні бути видимі для сторони яка займається їх обробкою.

\subsection*{Оцінка сучасного стану об’єкта дослідження або розробки}
Вперше технологія FHE була запропонована в 1978 в році, але майже 30 років не було
авторитетних досліджень на цю тему і на той момент вже існувала система RSA, яка була
краща за багатьма параметрам. Починаючи з 2009 року дослідження та розробки на
тему гомоморфного шифрування дуже актуальні й розвиток цієї технології відбувається
надзвичайно швидко, покращуючи швидкість виконання операцій, швидкість дешифрування, та
розширюючи область застосування шляхом додавання більш комплексних гомоморфних операцій.
На цей час існує багато рішень на типові проблеми з використовуванням FHE, які
конкурують між собою в різних аспектах, та постійно розвиваються.

\subsection*{Актуальність роботи та підстави для її виконання}
Безпечність передачі даних в незахищених середовищах та авторизація отримувача були 
завжди дуже важливими, рідко хто нехтує цим, оскільки не хоче, щоб їх данні були
скомпрометовані або перехоплені. Окрім цього все частіше, за потребою складних
обчислень, користувачі звертаються до віддалених машин, також відомі як хмари. Звісно
кожен користувач хоче, щоб їх данні були захищенні під час передачі, та хоче бути
впевнений, що він передає данні саме туди, куди планував.

Для забезпечення вище описаних вимог, користувач використовує чинні технології, такі
як PKI. Єдина не вирішена проблема PKI або інших технологій, це вимога повного
дешифрування даних, це означає що при отриманні зловмисником доступу до хмари або
віддаленого сервера, у нього буде доступ до не зашифрованих даних. Хоча сучасні хмари
дуже добре захищенні, розраховувати на те що зловмисник не зможе отримати до них
доступ - не варто. 

Для розв'язання проблеми, яку не вирішує PKI, чудово підходить FHE, оскільки сервер,
зберігає і виконує операції над даними в зашифрованому вигляді, тому навіть якщо
зловмисник отримає доступ до сервера або хмари, отримати дані в нього не вдасться.

Звісно є і деякі обмеження у використанні FHE: по-перше, операції над даними
обов'язково повинні бути гомоморфні, по-друге, алгоритм застосування операції над
зашифрованими даними дуже повільний. Якщо задача вимагає обробку великої кількості
даних, або операція повинна бути не гомоморфна, то можливо краще подумати в сторону
застосування інших криптосистем.

\subsection*{Мета й завдання роботи}
Написати пізніше
\subsection*{Об’єкт і методи дослідження}
об’єкт і методи дослідження або розроблення

\subsection*{Можливі сфери застосування}
Гомоморфне шифрування може бути застосоване в будь-якій сфері де потрібна обробка
даних, та для виконання цієї задачі використовується віддалений сервер, або хмара.
Використання FHE, гарантує безпечну передачу та обробку без попереднього дешифрування
даних, але при цьому накладає обмеження на операцію обробки, яка повинна бути
гомоморфна, та значно знижує час обробки.
