\newpage
\chapter*{скорочення та умовні позначення}
\addcontentsline{toc}{chapter}{\textsc{скорочення та умовні позначення}}

\textbf{\textsc{FHE}} -- Fully homomorphic encryption (Повне гомоморфне шифрування).

\textbf{\textsc{SHE}} -- Somewhat homomorphic encryption scheme

\textbf{\textsc{RSA}} -- Криптографічний алгоритм з відкритим ключем, який базується,
розрахунковій складності великих полупростих чисел.

\textbf{\textsc{PKI}} -- Public key infrastructure - набір інструментів які
використовують пару (приватний, публічний) ключ, та в якій між користувачами
передається тільки публічні ключі, залишаючи приватні анонімними.

\textbf{\textsc{Boolean Circuit}} -- Булева схема - це математична модель, що
використовується для представлення та обробки булевих функцій. Вона складається з логічних
елементів, які з'єднані між собою для виконання логічних операцій над двійковими входами
\(\{0,1\}\) та формування двійкових виходів. Схема складається з взаємопов'язаних логічних
елементів, таких як (AND, NOT, OR). 

\textbf{\textsc{FLT}} -- Мала теорема ферма \cite{Fermat}.

\textbf{\textsc{NAND gate}} -- NOT, AND операції за допомогою який може бути представлена
булева схема.
