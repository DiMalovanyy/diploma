\addcontentsline{toc}{chapter}{\textsc{додатки}}

\appendix

\chapter{Імплементація алгоритму пошуку в зашифрованій базі даних}
\label{appendix:a}
\small
\begin{lstlisting}[language=c++, tabsize=1]
std::optional<EncryptedDatabase::EncryptedEntry> 
    EncryptedDatabase::Lookup(const helib::Ctxt& key) const {
    std::vector<helib::Ctxt> ciphertextMask;
    ciphertextMask.reserve(encryptedKeyValueDb_.size());

    for(const auto& [encryptedKey, encryptedEntry]: 
        encryptedKeyValueDb_) {
        helib::Ctxt ciphertextMaskEntry = encryptedKey.key_;
        ciphertextMaskEntry -= key; 
        ciphertextMaskEntry.power(encryptedKey.plaintextPrimeModulus_ - 1);
        ciphertextMaskEntry.negate();
        ciphertextMaskEntry.addConstant(NTL::ZZX(1));
        std::vector<helib::Ctxt> rotatedCiphertextMasks(
            encryptedKey.encryptedArray.size(), ciphertextMaskEntry);
        for(int i = 1; i < rotatedCiphertextMasks.size(); i++) {
            encryptedArray.rotate(rotatedCiphertextMasks[i], i);
        }
        totalProduct(mask_entry, rotatedCiphertextMasks);
        mask_entry.multiplyBy(encryptedEntry);
        ciphertextMask.push_back(mask_entry);
    }

    helib::Ctxt value = ciphertextMask[0];
    for(int i = 1; i < ciphertextMask.size(); i++) {
        value += ciphertextMask[i];
    }

    return std::make_optional<EncryptedEntry>{value};
}
\end{lstlisting}
\par

\chapter{Створення контексту та приватного ключа}
\label{appendix:b}
\small
\begin{lstlisting}[language=c++, tabsize=1]
    // Plaintext prime modulus.
    long p = 2;
    // Cyclotomic polynomial - defines phi(m).
    long m = 4095;
    // Hensel lifting (default = 1).
    long r = 1;
    // Number of bits of the modulus chain.
    long bits = 500;
    // Number of columns of Key-Switching matrix (typically 2 or 3).
    long c = 2;
    // Factorisation of m required for bootstrapping.
    std::vector<long> mvec = {7, 5, 9, 13};
    // Generating set of Zm* group.
    std::vector<long> gens = {2341, 3277, 911};
    // Orders of the previous generators.
    std::vector<long> ords = {6, 4, 6};

    helib::Context context = helib::ContextBuilder<helib::BGV>()
                               .m(m)
                               .p(p)
                               .r(r)
                               .gens(gens)
                               .ords(ords)
                               .bits(bits)
                               .c(c)
                               .bootstrappable(true)
                               .mvec(mvec)
                               .build();
    // Create a secret key associated with the context.
    helib::SecKey secret_key(context);
    // Generate the secret key.
    secret_key.GenSecKey();
    // Generate bootstrapping data.
    secret_key.genRecryptData();
    // Public key management.
    // Set the secret key (upcast: SecKey is a subclass of PubKey).
    const helib::PubKey& public_key = secret_key;
    // Get the EncryptedArray of the context.
    const helib::EncryptedArray& ea = context.getEA();
    // Build the unpack slot encoding.
    std::vector<helib::zzX> unpackSlotEncoding;
    buildUnpackSlotEncoding(unpackSlotEncoding, ea);
\end{lstlisting}
\par

