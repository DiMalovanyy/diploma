\newpage
\chapter*{реферат}
Обсяг роботи 53 сторінки, 8 зображень, 5 лістингів, 33 джерел посилань.
ШИФРУВАННЯ, ГОМОМОРФНЕ ШИФРУВАННЯ, ХМАРНІ ТА ТУМАННІ КОМУНІКАЦІЇ, БЕЗПЕКА ПЕРЕДАЧІ ДАНИХ,
ЗАХИСТ ДАНИХ, БЕЗПЕКА, БЕЗПЕКА ДАНИХ В БАНКІВСЬКІЙ СИСТЕМІ.

Об'єктом роботи є дослідження можливостей використання гомоморфного шифрування в хмарних
та туманних обчисленнях. Предметом роботи, є реалізація спрощеної банківської системи, для
демонстрації можливостей гомоморфного шифрування.

Метою роботи є дослідження технології повного гомоморфного шифрування в хмарних та 
туманних технологій.

Методи розроблення: дослідження гомоморфних схем, аналітичне дослідження алгоритмів над
схемою. Інструменти розроблення: Мова С++, Бібліотека HeLib з вільно поширюваною ліцензією
Apache 2.0, додаткові бібліотеки для зручної роботи з Json, обробки та ініціації TCP 
з'єднань, та інш. 

Результати роботи: Описана логіка криптографічних схем гомоморфного шифрування, проведений
аналітичний огляд існуючих реалізацій схем, наведені переваги та недоліки використання 
технології гомоморфного шифрування, реалізований та продемонстрований в роботі застосунок
спрощеної банківської системи з використанням FHE.

Технологія гомоморфного шифрування, може використовуватись в будь-якій сфері де потрібна
конфіденційність даних, зокрема вона дозволяє тримати їх приватними для сторони яка їх
обробляє, що забезпечує ще вищий рівень безпеки.
